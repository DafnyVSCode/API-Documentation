\subsection{New Features} \label{featureExtensions}
This section details some interesting ideas that were gathered during development of the plugin, but were sadly out of scope of this project. It is thought as a starting point for developers that want to extend this plugin. 
\subsubsection{Debugger}
Integrating a debugger into the plugin would be great, since this is a feature that is very often used and very helpful when searching for bugs. An existing solution already support this feature, namely the Visual Studio integration \cite{visualstudiodafny} of Dafny. This already works very well and can be used as a guideline when developing an own integration. It is also a feature that programmers have become used to in all modern IDEs so it should be part of any language integration. \newline
Sadly it is also quite a difficult task since the interaction between an IDE, an executable and a debugger is complex. While the Visual Studio integration is open source, it is written in C\# and can therefore interface directly with the Dafny pipeline, which is heavily done in that integration. It would take quite some work to abstract this direct interaction into a clean API that extends the existing API of the Dafny pipeline. Using this API, a language server could provide an own implementation of a debugger.\newline
Designing this complex component was well out of scope for this project. It is estimated that, depending on the preexisting knowledge, this would be an endeavor of about two to four weeks time and probably would have to rely on at least some help by the Dafny development team. While this task is difficult, the result would be having an important feature that all users of modern IDEs anticipate in a language integration. It therefor should be a primary consideration when deciding on how to extend this project. 
\subsubsection{Widening Scope}
While many features have been implemented during the course of this project, the scope of their application was often narrowed as to provide a pragmatic approach to the problems regarding the scope of the project. For example, the rename element refactoring only works on members of classes, but not on parameters of methods. \newline
The exact scope of all features can be found in the API documentation of this project. While the extension of scope of existing features might seem tedious and not very interesting, it can introduce subtle problems that have to be dealt with with great care. It also is benefical to the users, as they can apply features in much wider contexts. 
\subsubsection{Contract Generation}
While contract generation has been implemented for some often occurring situations as detailed in \ref{dffeatures}, there is still a big potential for improvement in this area. Since a generic approach to contract generation has been deemed to be unfeasible, the next best way is to offer help in writing specification constructs in specific situations.\newline
When deciding to do work in this area, it is important to first analyze which situations appear often in a typical Dafny program. While this task it is difficult in itself, since it requires some expertise in Dafny, it is important because otherwise features may be implemented that do not get used in every day scenarios. \newline
When such situations have been identified, the specification construct generation must absolutely be correct. Since this is a feature that changes existing code, it is better to not provide any help when there is ambiguity regarding the correct solution. However, when done correctly, this is a feature from which programmers can greatly benefit, since it enhances productivity by eliminating the need to do complicated reasoning. It also introduces capabilities of Dafny to programmers that are not yet proficient in their usage of the language.
