\section{Testing}
This section details what was tested in this project and how the testing environment was set up.

\subsection{Test specification}

\subsubsection{Unit Tests}
Visual Studio Code plugins can be tested with standard javascript testing frameworks like Mocha or Jasmine, if they don't use any features of Visual Studio Code itself. This allows to test core logic without the need to run tests inside VS Code.

\subsubsection{Integration Tests}
Visual Studio Code extensions which require VS Code API can be tested with help of a special instance of Visual Studio Code. Inside this instance, the extension can use the full API and tests can execute commands like creating a new file, entering a character, using autocompletion and verifying if the extension works as expected.   

\subsubsection{Test overview}
To verify that the Dafny Visual Studio Code Plugin is working as expected, it has to be properly tested. Two possibilities exist. Either automatic unit / integration tests or manual testing. The best would be to test everything automatically. Unfortunately this is not possible, since although the Visual Studio Code Test API provides a rich set of methods, there is no direct access to the file system or a possibility to wait for certain tasks to finish. For this reason some tests have to be performed manually, especially use cases which install Dafny and rely heavily on scripts. On the other side, if all automatic tests fail, it's mostly a clear signal that something with the server is not okay. So the automatic tests also verify if the DafnyServer is working correctly. During development it is also quite easy to verify certain features on the fly. 
Below one first finds what is tested and how, followed by a table which explains how the individual use cases are tested and guides for manual testing. 

\subsubsection{System Tests}
As discussed in the last chapter, not everything can be easily tested, so the test scope has to be restricted. The following table shows which system component are tested and how.
\begin{longtable}{ p{0.3\textwidth} | p{0.2\textwidth} | p{0.4\textwidth} }
	\textbf{Topic} & \textbf{Tested} & \textbf{Description}\\
	Installation of Dafny plugin & Manual & Manual installation of Dafny and setting of base path   \\
	Automatic installation of Dafny & Manual & Easier to validate that base path is set correctly and all files exists \\
	Uninstallation of Dafny & Manual & Verify on file system level\\
	Communication DafnyServer and Dafny plugin & Manual + Automatic & Test if output from DafnyServer is correctly parsed and displayed\\
	Restart DafnyServer task & Manual & Check if a new PID is used and the other is stopped\\
	Local queue & Manual & DafnyServer can be paused in Visual Studio Code and queuing can be checked \\
	Syntax highlighting & Manual & Look on our examples \\
	Verification on typing & Manual & Easy to test manually\\
	Snippets & Manual & Check if snippets are working\\
	Document symbol provider & Automatic & Can easily be tested with the Visual Studio Code API \\
	Hover Provider & Automatic & Can easily be tested with the Visual Studio Code API\\
	Definition Provider & Automatic & Can easily be tested with the Visual Studio Code API  \\
	Code Lens & Automatic & Can easily be tested with the Visual Studio Code API\\
	Automatic generation of contracts & Manual & \todo{add desc} \\
	Autocompletion for identifiers & Automatic & \todo{add desc} \\
	Reporting of Dafny best practices violations & Manual & \todo{add desc} \\
\end{longtable}


\subsubsection{Basic tests}
If the Dafny plugin is installed, the following procedure is expected inside Visual Studio Code. 
\begin{itemize}
	\item The Dafny language is available
	\item .dfy is associated with the Dafny plugin
	\item Asks to install Dafny, if no DafnyServer is found 
	\item Starts the DafnyServer if a .dfy is opened
	\item Verification status of a .dfy file is shown
	\item Assertion errors are reported
\end{itemize}

\subsubsection{Detailed manual tests}
The following paragraphs explain how a certain feature can be tested manually, which steps have to be performed and what the expected result should be.

\paragraph{Manual installation}
\textbf{\newline Steps:}
\begin{enumerate}
	\item Uninstall the plugin and remove Dafny (/home/.Dafny or \%AppData\%/Roaming/Dafny)
	\item Remove the entry "dafny.basePath" from the user configuration file 
	\item Restart Visual Studio Code - There should be an error that Dafny is not installed
	\item Install Dafny from \href{https://github.com/FunctionalCorrectness/dafny-microsoft/releases}{https://github.com/FunctionalCorrectness/dafny-microsoft/releases}
	\item Set "dafny.basePath" to the extracted archive location
	\item Open the file simpleTest.dfy
	\item Execute "Dafny: RestartServer" task
\end{enumerate}
\textbf{\newline Expected:}
The DafnyServer should be started and the file verified. This can be seen in the status bar. On the right side, there should be the PID of the DafnyServer and on the left side the status "Verified"

\paragraph{Status change}
\textbf{\newline Steps:}
\begin{enumerate}
	\item Open the file invalidAsserts.dfy
	
\end{enumerate}
\textbf{\newline Expected:}
The server should first start and afterwards the status should display "Verifiying". After the DafnyServer verified the program, the status should have been changed to "Not Verified" and multiple asserts are marked red. 

\paragraph{Server crash}
\textbf{\newline Steps:}
\begin{enumerate}
	\item Open the file simpleTest.dfy
	\item The server should be started and the file verified
	\item Kill the DafnyServer process
\end{enumerate}
\textbf{\newline Expected:}
The server should restart automatically, with a corresponding message shown inside Visual Studio Code. 


\paragraph{Local queue}
\textbf{\newline Steps:}
\begin{enumerate}
	\item Open the file simpleTest.dfy
	\item The server should be started and the file verified
	\item Open Visual Studio Code and attach the debugger to the DafnyServer process
	\item Set a breakpoint in the verify method
	\item Change a assert in the simpleTest.dfy 
	\item Visual Studio Code should break at the breakpoint
	\item Open more files
\end{enumerate}
\textbf{\newline Expected:}
The queue size should be getting bigger and bigger as new files are opened. After the breakpoint is removed and the debugger detached, the queue size should decreasing. 

\paragraph{Verification on typing}
\textbf{\newline Steps:}
\begin{enumerate}
	\item Open the file simpleTest.dfy
	\item Change a assert that it does not hold (in under 0.7 seconds)
\end{enumerate}
\textbf{\newline Expected:}
After 0.7 seconds (default value) the file should be verified and the error should be displayed 

\paragraph{Automatic installation}
\textbf{\newline Steps:}
\begin{enumerate}
	\item Uninstall the Server with "Dafny: Uninstall DafnyServer" task
	\item Restart Visual Studio Code - There should be an error that Dafny is not installed
	\item Open the file simpleTest.dfy
	\item Install Dafny with the task "Dafny: Install DafnyServer" 
\end{enumerate}
\textbf{\newline Expected:}
Dafny should be downloaded, extracted, started and the file verified. See UC1, UC2, UC3 Expected


\paragraph{Uninstallation}
\textbf{\newline Steps:}
\begin{enumerate}
	\item Uninstall the Server with "Dafny: Uninstall DafnyServer" task
\end{enumerate}
\textbf{\newline Expected:}
Dafny should be uninstalled from the system. /home/.Dafny or \%AppData\%/Roaming/Dafny should be empty.


\paragraph{Snippets}
\textbf{\newline Steps:}
\begin{enumerate}
	\item Open the source code file snippets/dafny.json and remember a prefix from the list
	\item Enter the prefix 
	\item Press Enter
\end{enumerate}
\textbf{\newline Expected:}
The snippet should be inserted and can be edited


\paragraph{Autocompletion for identifiers}
Autocompletion can be tested with the Visual Studio Code Test API. For this, a file is loaded with a set of methods. Afterwards, some test check if the existing methods are considered for autocompletion. Additionally, more refactoring tests will be performed under this use case. 
