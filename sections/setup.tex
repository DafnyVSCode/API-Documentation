\section{System Setup}
This chapter details the infrastructure that was used in this project. It includes the local setup on the developer machines as well as all remote infrastructure such as continuous integration and project management tools. 
\subsection{Local setup}
\begin{itemize}
	\item TexStudio 2.12
	\item Visual Studio Code 1.9
	\item Git
\end{itemize}
\subsection{Server setup}
\begin{itemize}
	\item Jira
	\item Bamboo
	\item NodeJS
	\item SonarQube
	\item Postgres
\end{itemize}
\subsection{Continuous Integration}
To ensure that the code quality is high and the code is working as expected, all commits trigger automated builds of the respective project. A commit to the documentation repository executes a job which creates a file out of the \LaTeX sources. This document is then copied to the \emph{wwwroot} directory, which then can be downloaded from the project homepage. Commits to the vscode-dafny repository will result in a build and a complete run of all tests on the three environments afterwards. Therefore remote agents are installed on Ubuntu and OSX , which test the plugin on these operation system.  Additionally SonarQube is used to find bugs and bad practices as early as possible. The last repository which triggers a bamboo job is the project homepage. The latest version is built and deployed when a commit is performed. 
\begin{figure}[H]
	\centering
	\includegraphics[width=0.9\textwidth]{img/ci}
	\caption{Server Setup}
	\label{fig:Server setup}
\end{figure}