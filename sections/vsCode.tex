\section{Visual Studio Code}

\subsection{Visual Studio Code Plugin}
\subsubsection{Structure}
When a Visual Studio Code plugin is implemented as a language server, the plugin consists of two parts. The first one is an IDE-agnostic language server, which is further explained in \ref{langserver}. The second part is the actual plugin, which interacts with Visual Studio Code and relies on the language server to provide functionality. \newline
The actual plugin will further on be referred to as the client part of the plugin. The most important part of the client part is a file called package.json. It informs the IDE about the following points:
\paragraph{Activation}
Details for what the plugin is designed and when it is activated. Normally, plugins stay dormant until a use case is triggered for which they are needed. For this project, this is the case whenever a Dafny file is opened in Visual Studio Code.

\paragraph{Commands}
Details which commands this plugin exposes. Next to standard callbacks such as when a file is first opened or saved, additional custom commands can be registered here. This projects exposes some custom commands such as one to install the Dafny server or check the version of the server currently installed. This also includes key bindings for the registered commands.

\paragraph{Configuration}
Is used to configure a plugin. For this project this for instance entails the path to the Dafny executable or if it is running on a .NET or mono environment. It also sets Visual Studio Code specific configuration, such as how long of a delay should follow typing before the file is reevaluated.

\paragraph{Dependencies}
Details all the dependencies in the form of node modules that the plugin has.

\paragraph{Meta information}
In addition, this section details meta information about the plugin such as who developed it, which version it has and so on.

\subsubsection{Visual Studio Code Customization}
Next to the above mentioned part, the client part of the plugin is needed to instantiate the server part of the plugin and relaying the requests by the IDE to the server and the responses from the server to the IDE. This is done in a few lines of code, since Visual Studio Code fully implements the language server protocol. The instantiation and relaying of messaging is therefore done by simply registering the correct components to handle the correct requests. Details on which components are registered high can be found in \ref{langserver} and \ref{architecture}.\newline
In addition, the client part also handles GUI customization. This is needed to correctly represent the intentions of the custom messages that are defined in \ref{custom commands}. It entails features such as displaying a progress bar while the Dafny pipeline is downloaded during installation and updating it accordingly. Since the messages are custom, there is no predefined way on how the GUI should deal with them. Here a minimal working knowledge of the IDE in question is needed in order to decide how certain messages should be represented in the GUI, since some idioms have been established over the years by every IDE. \newline
This part of the client is also the only part that has to be written separately for every IDE integration of the language server. However, since it consists of about 100 lines of code, this should be doable in a short amount of time. This concludes the client part of the plugin, it can be seen that this part is neglectable in terms of size and complexity compared to the rest of the implementation developed during this project. 

